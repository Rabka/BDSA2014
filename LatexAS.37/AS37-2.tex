\documentclass{article}

\usepackage{amsmath}
\usepackage{amssymb}
\usepackage{graphicx}

% For at få faver på texten skal man gå til "Format" og så "Syntax Colering".
% Her kom jeg til https://www.youtube.com/watch?v=Xl1c_1yL47k

\begin{document}

	\title{"Assignment 37"}
	\author{"pebj, smot"}
	\maketitle

	\begin{enumerate}
		\item Identify all relevant \textbf{Actors}.
		\begin{enumerate}
			\item[Nr.1] Client
			\item[Nr.2] Server
		\end{enumerate}
		\item Identify and describe \textbf{2 non-trivial Scenarios}.
	\end{enumerate}

\begin{tabular}{c r @{} l}
	\multicolumn{2}{c}{} \\
	\hline
	Scenario name:	&&Entry sharing\\
	\hline
	Praticipating:		&&Claus, jan: Client \\
	\hline
	Flow of events:	&1)&Claus' birthday is coming up, so he needs to invite any participants\\ 
					&&who would like to come.\\
				&2)&Claus therefore makes an entry in the CALENDAR which gives him a link\\ 
					&&that he can post on his wall on Facebook etc.\\
				&3)&The friend Jan notices this link and wishes to take part in Claus' birthday.\\ 
					&&He therefore visits the link which puts this event in his calendar.\\
				&4)&Ocautionary Jan will visit and check his calender and see that the entry\\ 
					&&in the calendar made by Claus is shown at the correct date. \\
	\hline
\end{tabular}
\\


\begin{tabular}{c r @{} l}
	\multicolumn{2}{c}{} \\
	\hline
	Scenario name:	&&Upcomming event notification\\
	\hline
	Praticipating:		&&Claus, jan: Client \\
	\hline
	Flow of events:	&1)&A week ago Claus inserted an entry to his CALENDAR where he stated\\ 
					&&that tomorow from the day he would attend an important meeting.\\
				&2)&Claus now hears an email notification sound from his smartphone\\ 
					&&which he then reads.\\
				&3)&The Email tells him of an upcomming event, namely the important meeting.\\ 
					&&Claus is now calm, he has been reminded of something he almost forgot.\\
				&4)&The next day, Claus attends the meeting.\\
	\hline
\end{tabular}




	\begin{enumerate}
		\item[3.] Identify all  relevant \textbf{Use Cases} and draw the "sticky man" UML diagram.

	\end{enumerate}

\end{document}